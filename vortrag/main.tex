%!TeX engine = xelatex
\def\BeamerAspectRatio{43}

\documentclass[aspectratio=\BeamerAspectRatio]{beamer}

\usepackage{Packages}

\import{./}{TitlePageInfo.tex}
\title[Dezibot Schwarm Logging]{Dezibot Schwarm Logging \\ \& Monitoring}
\subtitle[Projektvortrag]{Vom Einzelbot-Debugging zur Schwarmplattform}
\author{Projektteam Dezibot: Niclas Jost, Marius Busalt}
\renewcommand{\faculty}{Fakult\"at Informatik und Medien}
\date{Projektpr\"asentation, Februar 2026}

\begin{document}
\setcounter{tocdepth}{1}
\maketitlepage
\maketableofcontents

\section{Motivation}

\subsection{Stand Vorgängerprojekt}
\begin{frame}{Ausgangslage vor diesem Projekt}
	\begin{itemize}
		 \item Debug-Server zeigte nur die Sensordaten des Webserver-Host Dezibots
   		 \item Websiten mit akutellen Sensordaten und Logs
   		 \item Fokus lag auf lokaler Diagnose eines einzelnen Dezibots statt auf koordinierter Schwarmüberwachung
		 \item Keine Kommunikation zu anderen Dezibots implementiert
		 \item Hauptloop blockierte Sensordatenabfrage und führte zu Datenverlust
	\end{itemize}
\end{frame}

\subsection{Motivation}
\begin{frame}{Ursprüngliche Verbesserungsideen}
	\begin{itemize}
		\item Überwachung einer Dezibotgruppe bzw. Schwarm
		\item Schnelle Übersicht über Schwarm: Wer ist online, wer sendet, Logs einsehen
		\item Fokus auf Diagnostik, ausgelesene Sensordaten einsehbar und spezifische Logs verfügbar machen
		\item Identifizierung von Dezibots ermöglichen, um Daten leichter zuordnen zu können		
	\end{itemize}
\end{frame}

\subsection{Was waren unsere Ziele}
\begin{frame}{Projektziele}
	\begin{itemize}
		\item Zentrale Schwarm-Übersicht aller verbundenen Dezibots über Masterdezibot
		\item Telemetrie und Logs einzelner Dezibots verfügbar machen
		\item Bidirektionale Kommunikation: Masterdezibot soll Telemetrie empfangen und Kommandos zurücksenden
		\item Zwei Kommunikationsprotokolle für verschiedene Usecases: ESP-NOW und Bluetooth Low Energy
		\item Stabiles Datenmanagment und Echtzeitverfügbarkeit durch entkoppeltes Telemetriemanagment
		\item Verbesserte UI passend zu neuen Funktionen, Speicherverbrauch reduzieren
		\item Logging für größere Schwärme ermöglichen
	\end{itemize}
\end{frame}

\section{Grundlegende Architektur}

\subsection{Plantuml Grobarchitektur}
\begin{frame}{PlantUML Grobarchitektur (Systemkontext)}
	\centering
	\includegraphics[width=\textwidth,height=0.75\textheight,keepaspectratio]{imglib/diagrams/grobarchitektur.png}
\end{frame}

\subsection{Plantuml Klassendiagramm}
\begin{frame}{PlantUML Klassendiagramm (Transport-Layer)}
	\centering
	\includegraphics[width=\textwidth,height=0.75\textheight,keepaspectratio]{imglib/diagrams/klassendiagramm.png}
\end{frame}

\subsection{Dev Experience mit Platform IO}
\begin{frame}{Dev Experience mit PlatformIO}
	\begin{itemize}
		\item TODO rausnehmen?!
		\item Einheitliche Build-Umgebung für Sender und Empfanger,  gesamte Config in platform.ini einsehbar
		\item Reproduzierbarer Ablauf: \texttt{pio run}, \texttt{upload}, \texttt{uploadfs}, \texttt{device monitor}.
		\item Frontend-Artefakte können optional in \texttt{data/} gebaut und per SPIFFS hochgeladen werden.
		\item Kernnutzen bleibt die schnelle Iteration an Firmware und Kommunikationslogik.
	\end{itemize}
\end{frame}

\begin{frame}[fragile]{Build- und Flash-Workflow}
	\small
	\item TODO rausnehmen?
	\begin{verbatim}

cd web
npm install
npm run build

pio run -e esp32s3_receiver -t uploadfs
pio run -e esp32s3_receiver -t upload
pio run -e esp32s3_sender -t upload
\end{verbatim}
	\begin{itemize}
		\item Ein Build erzeugt sofort testbare Firmware und die passende Weboberflache.
	\end{itemize}
\end{frame}

\section{Funktionen}

\subsection{Übersicht verbundener Dezibots}
\begin{frame}{Schwarm-Übersicht im Dashboard}
	\begin{itemize}
		\item TODO überarbeiten
		\item Je Bot sichtbar: Online/Offline, MAC, Message Counter, Uptime, Last Seen
		\item Klick auf einen Bot führt direkt in die Live-Sensoransicht für diesen Dezibot
		\item Locate-Button lässt Dezibot blinken und macht diesen sofort identifzierbar
	\end{itemize}
\end{frame}

\begin{frame}{Charts mit Live-Sensordaten}
	\begin{itemize}
		\item Pro Sensorwert ein eigenes Echtzeit-Diagram 
		\item Anzeige umfasst Sensordaten und Systemmetriken wie Heapinfo oder Chip-Temperatur
		\item Zeitfenster ist konfigurierbar 50 bis 2000 Datenpunkte für Debugging und Demo
		\item Fokus auf Lesbarkeit bei mehreren Bots statt auf historische Langzeitarchivierung
	\end{itemize}
\end{frame}

\begin{frame}{Logging}
	\begin{itemize}
		\item Eigene Logging-Seite mit Filterung nach Level (ALL, INFO, WARNING, ERROR, DEBUG, TRACE)
		\item Polling-basierter Nachlade-Mechanismus für neue Einträge in nahezu Echtzeit
		\item Ringpuffer im Backend verhindert unkontrolliertes Speicherwachstum
		\item Ermöglicht wichtigen Debugging Einstieg
	\end{itemize}
\end{frame}

\subsection{Bidirektionale Kommunikation}
\begin{frame}{Kommunikationsprotokolle}
	\begin{itemize}
		\item TODO kann man auf einem Dezibot BLE und auf dem anderes ESP now laufen lassen udn parallel Daten auslesen?
		\item ESP-NOW und BLE GATT implementiert
		\item Transport-Adapter Pattern Prozess und Kommunikationslogik getrennt
		\item  -> Flexibilität um verschieden Usecases abzudecken
		\item ESP-Now: 200-500m Range, bis zu 1 MBPS 
		\item BLE: 10-50m Range,  bis zu 2 MBPS
		\item Beide Prtokolle können theoretisch unendlich viele Dezibots verbinden
	\end{itemize}
\end{frame}

\begin{frame}{ESP-NOW: Telemetrie und Kommandos}
	\begin{itemize}
		\item TODO Magic Numbers durch Wert ersetzen
		\item Sender broadcastet \texttt{SensorMessage} im Sekundentakt, Empfänger sammelt zentral
		\item Kommandos gehen als Unicast zurück an ein konkretes Gerät, z.B. Locate Dezibot
		\item \texttt{Magic Numbers} trennen Sensor- und Kommando-Pakete robust
	\end{itemize}
\end{frame}

\begin{frame}{Bluetooth (BLE GATT)}
	\begin{itemize}
		\item Sender als GATT-Server, Empfänger als GATT-Client
		\item Sensordaten kommen per Notification, Kommandos per Write
		\item Ermöglicht gemischte Setups mit ESP-NOW und BLE innerhalb derselben Plattform
	\end{itemize}
\end{frame}

\subsection{Web-Technologien Build Chain auf Dezibot}
\begin{frame}{Web-Build-Chain (Randaspekt)}
	\begin{itemize}
		\item TODO stimmt die KB aussage?
		\item UI basiert auf SolidJS/Vite und liegt nur auf Serverseite
		\item Monitoring nutzerfreundlicher gestalten
		\item Nebeneffekt: JavaScript-Bundle sank von 507 KB auf ca. 376 KB.
	\end{itemize}
\end{frame}

\section{Probleme}

\begin{frame}{Technische Herausforderungen im Betrieb}
	\begin{itemize}
		\item TODO Würde ich löschen
		\item BLE-Skalierung: gleichzeitige Verbindungen auf ESP32-S3 sind begrenzt.
		\item BLE-Onboarding: Scan- und Verbindungsaufbau verursachen spurbare Verzogerung.
		\item Kanalabhangigkeit bei ESP-NOW: Sender und Empfanger mussen denselben Kanal nutzen.
		\item Blockierender Locate-Handler kann den Empfangspfad für einige Sekunden storen.
	\end{itemize}
\end{frame}

\begin{frame}{Qualitat und Grenzen der Messwerte}
	\begin{itemize}
		\item TODO Würde ich löschen
		\item Power-Wert ist bewusst eine Schatzung und keine elektrische Messung.
		\item Chip-Temperatur ist intern und nur begrenzt als Umgebungsindikator nutzbar.
		\item Thread-Safety in Teilen des Logging-Zugriffs ist als Verbesserungsaufgabe dokumentiert.
		\item Diese Punkte sind dokumentiert und priorisiert für die nachsten Iterationen.
	\end{itemize}
\end{frame}

\section{Ausblick}

\begin{frame}{Roadmap}
	\begin{itemize}
		\item Sensor-Export als CSV/JSON für Analyse in Python oder Tabellenkalkulation
		\item Device Naming für besser lesbare Bot-Identifikation anstatt reiner MAC-Adressen
		\item Erweiterte Remote-Commands für Motoren, LEDs und gruppierte Aktionen
		\item Alerting bei Schwellwerten, z. B. Heap, Temperatur oder Verbindungsverlust
		\item Funktionen wie OTA-Update, Fernsteuerung, RSSI-basierte Entfernungsschätzung oder Ladestandanzeige des Dezibots
	\end{itemize}
\end{frame}

\begin{frame}{Fazit}
	\begin{itemize}
		\item Aus einem Einzelbot-Debugging wurde eine zentrale Schwarmüberwachung entwickelt
		\item Architektur ist so aufgebaut, dass weitere Protokolle und Features anschlussfähig sind
		\item Mehrwert unserer Erweiterung ist unter anderem das Live-Monitoring anderer Bots, Daten- und Kommandokommunikation sowie Protokollauswahl
		\item Größere Schwärme überwachbar durch speichereffiziente UI und konsequentes Speichermanagment
	\end{itemize}
	\vspace{0.8em}
	\centering
	\Large Fragen?
\end{frame}

\begin{frame}{Quellen und Referenzen}
	\small
	\begin{itemize}
		\item Projekt-Repository und README: \url{https://github.com/dezibot/dezibot-Schwarm-logging}
		\item Dezibot Logging Vorgangerprojekt: \url{https://github.com/Tim-Dietrich/dezibot-logging}
		\item Dezibot4 Bibliothek: \url{https://github.com/dezibot/dezibot}
		\item ESP-NOW Dokumentation (Espressif): \url{https://docs.espressif.com/}
		\item TODO anhängen PlatformIO, SolidJS, Vite, Chart.js Projektdokumentationen
	\end{itemize}
\end{frame}

\end{document}
