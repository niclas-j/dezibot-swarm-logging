% !TeX program = xelatex
\def\BeamerAspectRatio{43}

\documentclass[aspectratio=\BeamerAspectRatio]{beamer}

\usepackage{Packages}

\import{./}{TitlePageInfo.tex}
\title[Dezibot Schwarm Logging]{Dezibot Schwarm Logging \\ \& Monitoring}
\subtitle[Projektvortrag]{Vom Einzelbot-Debugging zur Schwarmplattform}
\author{Projektteam Dezibot: Niclas Jost, Marius Busalt}
\renewcommand{\faculty}{Fakult\"at Informatik und Medien}
\date{Projektpr\"asentation, Februar 2026}

\begin{document}
\setcounter{tocdepth}{1}
\maketitlepage
\maketableofcontents

% 1-5 Marius
% 6-10 Niclas
% 11,12,13,14 Marius
% 15 - Rest

\section{Motivation}

\subsection{Stand Vorgängerprojekt}
% 
\begin{frame}{Ausgangslage vor diesem Projekt}
	\begin{itemize}
		\item Debug-Server zeigte nur die Sensordaten des Webserver-Host Dezibots
		\item Es gab nur Diagnose eines einzelnen Dezibots
		\item Keine Kommunikation zu anderen Dezibots implementiert
		\item Großes Webbundle
	\end{itemize}
\end{frame}

\subsection{Motivation}
\begin{frame}{Ursprüngliche Verbesserungsideen}
	\begin{itemize}
		\item Überwachung einer Dezibotgruppe bzw. Schwarm
		\item Schnelle Übersicht über Schwarm: Wer ist online, wer sendet, Logs einsehen
		\item Fokus auf Diagnostik, ausgelesene Sensordaten einsehbar und spezifische Logs verfügbar machen
		\item Identifizierung von Dezibots ermöglichen, um Daten leichter zuordnen zu können
	\end{itemize}
\end{frame}

\subsection{Was waren unsere Ziele}
\begin{frame}{Projektziele}
	\begin{itemize}
		\item Zentrale Schwarm-Übersicht aller verbundenen Dezibots über Masterdezibot
		\item Telemetrie und Logs einzelner Dezibots verfügbar machen
		\item Bidirektionale Kommunikation: Masterdezibot soll Telemetrie empfangen und Kommandos zurücksenden
		\item Kommunikationsprotokoll-Agnostisch
		\item Echtzeitverfügbarkeit durch entkoppeltes Telemetriemanagment
		\item Verbesserte UI passend zu neuen Funktionen, Speicherverbrauch reduzieren
		\item Logging für größere Schwärme ermöglichen
	\end{itemize}
\end{frame}

\section{Architektur}

\subsection{Plantuml Grobarchitektur}
\begin{frame}{PlantUML Grobarchitektur (Systemkontext)}
	\centering
	\includegraphics[width=\textwidth,height=0.75\textheight,keepaspectratio]{imglib/diagrams/grobarchitektur.png}
\end{frame}

\subsection{Plantuml Klassendiagramm}
\begin{frame}{PlantUML Klassendiagramm (Transport-Layer)}
	\centering
	\includegraphics[width=\textwidth,height=0.75\textheight,keepaspectratio]{imglib/diagrams/klassendiagramm.png}
\end{frame}


\begin{frame}[fragile]{Build- und Flash-Workflow}
	\begin{itemize}
		\item Einheitliche Build-Umgebung für Sender und Empfanger,  gesamte Config in platform.ini einsehbar
		\item Frontend-Artefakte können optional in \texttt{data/} gebaut und per SPIFFS hochgeladen werden.
	\end{itemize}
	\begin{verbatim}

cd web
npm install
npm run build

pio run -e esp32s3_receiver -t uploadfs
pio run -e esp32s3_receiver -t upload
pio run -e esp32s3_sender -t upload
\end{verbatim}

\end{frame}

\section{Funktionen}

\subsection{Übersicht verbundener Dezibots}
\begin{frame}{Schwarm-Übersicht im Dashboard}
	\begin{itemize}
		\item Je Bot sichtbar: Online/Offline, MAC, Message Counter, Uptime, Last Seen
		\item Klick auf einen Bot führt direkt in die Live-Sensoransicht für diesen Dezibot
		\item Senden von Commands durch verschiedene Buttons
	\end{itemize}
\end{frame}

\begin{frame}{Charts mit Live-Sensordaten}
	\begin{itemize}
		\item Pro Sensorwert ein eigenes Echtzeit-Diagram
		\item Anzeige umfasst Sensordaten und Systemmetriken wie Heapinfo oder Chip-Temperatur
		\item Anzahl der Datenpunkte ist konfigurierbar
	\end{itemize}
\end{frame}

\begin{frame}{Logging}
	\begin{itemize}
		\item Eigene Logging-Seite mit Filterung nach Level (ALL, INFO, WARNING, ERROR, DEBUG, TRACE)
		\item Polling-basierter Nachlade-Mechanismus für neue Einträge in nahezu Echtzeit
		\item Ringpuffer im Receiver-Dezibot verhindert unkontrolliertes Speicherwachstum
		\item Ermöglicht wichtigen Debugging Einstieg
	\end{itemize}
\end{frame}

\subsection{Bidirektionale Kommunikation}
\begin{frame}{Kommunikationsprotokolle}
	\begin{itemize}
		\item ESP-NOW und BLE GATT implementiert
		\item Transport-Adapter Pattern Prozess und Kommunikationslogik getrennt
		\item  -> Flexibilität um verschieden Usecases abzudecken, erweiterbar
		\item ESP-Now: 200-500m Range, bis zu 1 MBPS
		\item BLE: 10-50m Range,  bis zu 2 MBPS
	\end{itemize}
\end{frame}

\begin{frame}{ESP-NOW: Telemetrie und Kommandos}
	\begin{itemize}
		\item Sender broadcastet \texttt{SensorMessage} im Sekundentakt, Empfänger sammelt zentral
		\item Kommandos gehen vom Dezibot-Receiver an einen spezifischen Dezibot-Sender, z.B. Locate Dezibot
		\item \texttt{Magic Numbers} trennen Sensor- und Kommando-Pakete robust
	\end{itemize}
\end{frame}

\begin{frame}{Bluetooth (BLE GATT)}
	\begin{itemize}
		\item Sender als GATT-Server, Empfänger als GATT-Client
		\item Sensordaten kommen per Notification, Kommandos per Write
		\item Ermöglicht gemischte Setups mit ESP-NOW und BLE innerhalb derselben Plattform
	\end{itemize}
\end{frame}

\subsection{Web-Technologien Build Chain auf Dezibot}
\begin{frame}{Web-Server}
	\begin{itemize}
		\item UI basiert auf SolidJS/Vite als Single-Page-Application
		\item Monitoring nutzerfreundlicher gestalten
		\item Bundle Größe wurde deutlich reduziert (530kb -> 135kb)
		\item Export-Funktionalität
	\end{itemize}
\end{frame}



\begin{frame}{Demo}
\end{frame}



\section{Ausblick}

\begin{frame}{Roadmap}
	\begin{itemize}
		\item Sensor-Export als CSV/JSON für Analyse in Python oder Tabellenkalkulation
		\item Device Naming für besser lesbare Bot-Identifikation anstatt reiner MAC-Adressen
		\item Erweiterte Remote-Commands für Motoren, LEDs und gruppierte Aktionen
		\item Alerting bei Schwellwerten, z. B. Heap, Temperatur oder Verbindungsverlust
		\item Funktionen wie OTA-Update, Fernsteuerung, RSSI-basierte Entfernungsschätzung oder Ladestandanzeige des Dezibots
	\end{itemize}
\end{frame}

\begin{frame}{Fazit}
	\begin{itemize}
		\item Aus einem Einzelbot-Debugging wurde eine zentrale Schwarmüberwachung entwickelt
		\item Architektur ist so aufgebaut, dass weitere Protokolle und Features anschlussfähig sind
		\item Mehrwert unserer Erweiterung ist unter anderem das Live-Monitoring anderer Bots, Daten- und Kommandokommunikation sowie Protokollauswahl
		\item Größere Schwärme überwachbar durch speichereffiziente UI und konsequentes Speichermanagment
	\end{itemize}
	\vspace{0.8em}
	\centering
	\Large Fragen?
\end{frame}

\begin{frame}{Quellen und Referenzen}
	\small
	\begin{itemize}
		\item Projekt-Repository und README: \url{https://github.com/niclas-j/dezibot-swarm-logging}
		\item Dezibot Logging Vorgangerprojekt: \url{https://github.com/Tim-Dietrich/dezibot-logging}
		\item Dezibot4 Bibliothek: \url{https://github.com/dezibot/dezibot}
		\item ESP-NOW Dokumentation (Espressif): \url{https://docs.espressif.com/}
	\end{itemize}
\end{frame}

\end{document}
